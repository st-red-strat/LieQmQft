\section{量子力学の基礎}

量子力学に特徴的な事は、
物理量が単なる数ではなくHilbert空間に作用する
非可換なoperatorとなる事である。
観測可能な量はHermitian operatorとなるので、
我々はoperatorとして専らlinearな
Hermitianないしunitary operatorを扱う。

\subsection{正準量子化}

\subsubsection{状態空間とOperator}

量子力学に現れるoperator $O$とは、
写像$O: \mathcal{H} \to \mathcal{H}$
すなわち
ある複素vector space $\mathcal{H}$の
元$\ket{\psi}$に作用して
再びvector space の元
$O\ket{\psi} \in \mathcal{H}$
を与えるものである。

あるoperator $O$が$\mathcal{H}$にlinearに作用している、
あるいはlinearである、とは
\begin{align}
    \text{ For }
        \forall \ket{\psi_1}, \ket{\psi_2}
        \in \mathcal{H}
    \text{ and }
        \forall a,b\in \mathbb{C}
    ,\qquad
        O\Big(
            a \ket{\psi_1}
            +
            b \ket{\psi_2}    
        \Big)
    =
        a O \ket{\psi_1}
        +
        b O \ket{\psi_2}    
\end{align}
であることを言う。
例えば時間反転操作に対応するoperator $T$は
anti-unitary (anti-linearかつunitary)
\begin{align}
    \text{ For }
        \forall \ket{\psi_1}, \ket{\psi_2}
        \in \mathcal{H}
    \text{ and }
        \forall a,b\in \mathbb{C}
    ,\qquad
        T\Big(
            a \ket{\psi_1}
            +
            b \ket{\psi_2}    
        \Big)
    =
        a^* T \ket{\psi_1}
        +
        b^* T \ket{\psi_2}    
\end{align}
なoperatorの重要な例であるが、
以下では専らlinearなものに話を限る。

量子力学では物理量はoperatorで表され、
解析力学で基本的な力学自由度であった
$\{q\},\{p\}$
さえもoperatorとなっている。
我々は任意の観測量の時間発展が決定論的な
物理法則によって記述されることを仮定するが、
一方で直接観測される量はもちろん実数であるので、
まずはこれらの観測量を
$\{\hat{q}\},\{\hat{p}\}$
のようなoperatorと関係付ける方法を考えなければならない。
以下ではこの方法を
Hilbert空間と呼ばれるvector spaceを用いて与えよう。

vector space $\mathcal{H}$が
内積$\cdot\ (\ ,\ )$を持つとは、
$
\forall \ket{\psi_1}, \ket{\psi_2}
\in \mathcal{H}
$
に対し複素数
$ \braket{ \psi_1 | \psi_2 }
:= \cdot\ (\ket{\psi_1}, \ket{\psi_2})
\in \mathbb{C} $
を与える写像
$\cdot\ (\ ,\ ): 
\mathcal{H}\times \mathcal{H}
\to \mathbb{C}$
であって、
\begin{align}
    \braket{ \psi | \phi } &= \braket{ \phi | \psi }^*
&\text{(共役対称性)}
\\
            \bra{\phi}\Big(
            a \ket{\psi_1}
        +
            b \ket{\psi_2}
        \Big)
    &=
        a \braket{\phi|\psi_1}
    +
        b \braket{\phi|\psi_2}
    \qquad
    \text{ for } \forall
    a,b\in \mathbb{C}
&\text{(線形性)}
\\
    \braket{ \psi | \psi } \ge 0
    ,\quad&\text{and}\quad
    \braket{ \phi | \phi } = 0
    \Leftrightarrow
    \ket{\phi}=0
&\text{(正定値性)}
\end{align}
を満たすものがあることを言う。
Hilbert空間とは内積空間であって完備な
(直感的には、極限が定義されている)
ものを言う。
物理学において時間発展は微分方程式で与えられるので、
微分を定義するために
極限が定義されている必要があるのである。

量子力学において決定論的な時間発展方程式に従う力学自由度は
Hilbert空間の元である。
このHilbert空間を状態空間と言い、
その元を状態vectorと呼ぶ。
個々の状態vectorの
時間発展はSchr\"odinger方程式
\begin{align}
    i\hbar \dfrac{d}{dt}
        \ket{\psi(t)}
    &=
    \hat{H} \ket{\psi(t)}
\end{align}
によって与えられる。
ここで$\hbar$は換算Planck定数またはDirac定数と呼ばれ、
Planck定数$h$により
\begin{align}
    \hbar := \dfrac{h}{2\pi}
\end{align}
と定義される。
また、
$\hat{H}$は以下で定義する
Hamiltonian operatorである。

\subsubsection{正準交換関係(CCR: Canonical Commutation Relation)}

二つのoperator $\hat{A}, \hat{B}$の間の交換関係を
\begin{align}
    [\hat{A}, \hat{B}] := \hat{A} \hat{B} - \hat{B} \hat{A}
\end{align}
で定義し、$[\hat{A}, \hat{B}] = 0$であるとき
$\hat{A}, \hat{B}$は可換であるという。
量子力学を考えるまで、
あらゆる量は可換であった。
このように全ての量が可換である力学系を
古典力学系と言い、そこに現れる可換な数を$c$-数という。

ある古典力学系のHamiltonian $H(\{q\},\{p\})$
が知られているとき、
その正準力学変数$\{q\},\{p\}$を
正準交換関係:
\begin{align}
    [ \hat{q}_i , \hat{p}_j ] = i\hbar \delta_{ij}
\end{align}
を満たすoperatorの組
$ \{\hat{q}\} , \{\hat{p}\} $
で置き換える手続きを正準量子化と呼び、
\begin{align}
    \hat{H}
    &:=
    H(\{\hat{q}\} , \{\hat{p}\})
\end{align}
を得られた量子力学系の
Hamiltonian operator
という。

\subsubsection{調和振動子の例}

調和振動子を正準量子化してみよう。
\begin{align}
    \hat{H} &= \sum_k \bigg(
        \dfrac{ \hat{p}_k^2 }{2m} 
    +
        \dfrac{ m\omega^2 }{2} \hat{q}_k^2
    \bigg)
\\
    [ \hat{q}_i , \hat{p}_j ] &= i\hbar \delta_{ij}
\end{align}
より、公式(\ref{A,BC to B(A,C) + (A,B)C})を使って
\begin{align}
    [q_i, H] &= i\hbar \dfrac{ \hat{p}_i }{m} 
\\
    [p_j, H] &= - i \hbar\ m\omega^2 \hat{q}_j
\end{align}
