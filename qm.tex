\section{量子力学の基礎}

量子力学に特徴的な事は、
物理量が単なる数ではなく非可換なoperatorとなる事である。
観測可能な量はHermitian operatorとなるので、
我々はoperatorとして専らHilbert空間に作用するlinearな
Hermitianないしunitary operatorを扱う。

\subsection{正準交換関係(CCR: Canonical Commutation Relation)}

二つのoperator $\hat{A}, \hat{B}$の間の交換関係を
\begin{align}
    [\hat{A}, \hat{B}] := \hat{A} \hat{B} - \hat{B} \hat{A}
\end{align}
で定義し、$[\hat{A}, \hat{B}] = 0$であるとき
$\hat{A}, \hat{B}$は可換であるという。
量子力学を考えるまで、
あらゆる量は可換であった。
このように全ての量が可換である力学系を
古典力学系と言い、そこに現れる可換な数を$c$-数という。

ある古典力学系のHamiltonianが知られているとき、
その正準力学変数$\{q\},\{p\}$を
\begin{align}
    [ \hat{q}_i , \hat{p}_j ] = i\hbar \delta_{ij}
\end{align}
を満たすような演算子の組
$ \{\hat{q}\} , \{\hat{p}\} $
で置き換える手続きを正準量子化という。

\subsubsection{調和振動子の例}

調和振動子を正準量子化してみよう。
\begin{align}
    \hat{H} &= \sum_k \bigg(
        \dfrac{ \hat{p}_k^2 }{2m} 
    +
        \dfrac{ m\omega^2 }{2} \hat{q}_k^2
    \bigg)
\\
    [ \hat{q}_i , \hat{p}_j ] &= i\hbar \delta_{ij}
\end{align}
より、公式(\ref{A,BC to B(A,C) + (A,B)C})を使って
\begin{align}
    [q_i, H] &= i\hbar \dfrac{ \hat{p}_i }{m} 
\\
    [p_j, H] &= - i \hbar\ m\omega^2 \hat{q}_j
\end{align}
