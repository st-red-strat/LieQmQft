\section{量子力学の基礎}

量子力学に特徴的な事は、
物理量が単なる数ではなく非可換なoperatorとなる事である。
観測可能な量はHermitian operatorとなるので、
我々はoperatorとして専らHilbert空間に作用するlinearな
Hermitianないしunitary operatorを扱う。

\subsection{正準交換関係(CCR: Canonical Commutation Relation)}

二つのoperator $\hat{A}, \hat{B}$の間の交換関係を
\begin{align}
    [\hat{A}, \hat{B}] := \hat{A} \hat{B} - \hat{B} \hat{A}
\end{align}
で定義し、$[\hat{A}, \hat{B}] = 0$であるとき
$\hat{A}, \hat{B}$は可換であるという。
量子力学を考えるまで、
あらゆる量は可換であった。
このように全ての量が可換である力学系を
古典力学系と言い、そこに現れる可換な数を$c$-数という。

ある古典力学系のHamiltonianが知られているとき、
その正準力学変数$\{q\},\{p\}$を
\begin{align}
    [ \hat{q_i} , \hat{p_j} ] = i\hbar \delta_{ij}
\end{align}
を満たすような演算子の組
$ \{\hat{q}\} , \{\hat{p}\} $
で置き換える手続きを正準量子化という。
