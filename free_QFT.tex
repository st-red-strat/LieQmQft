\section{自由場の量子論}
\label{sec: QFT}

場の量子論と言うとFeynman diagram
(古くはFeynman graphと呼ぶ文献もある)を使った計算に
皆さん憧れているのだろうが、
これは摂動計算に現れるものである。
摂動論は問題を解ける部分とそこからのズレに分けて
解析するというものであったので、
とにかく解ける理論がないと話にならない。
可解な場の量子論もそれ自身興味深い話題ではあるものの、
以下では解ける模型として非相対論的な場合から始め、
相対論的な場まで含めた自由場の量子化を扱う。

\subsection{自由場の古典論、場の解析力学}

\subsubsection{場のLagrangianとLagrange形式}

量子化するというからには
量子化される対象となる古典論を用意する必要がある。
空間$d$次元、時間$1$次元の$D:= d+1$次元時空$\mathcal{M}^{1,d}$を考え、
その座標を$(t, x^1,\dots, x^d)$とする。
時空$\mathcal{M}^{1,d}$上で定義された
適当な場$\phi(t, \bm{x})$と
その空間微分または時間微分
\begin{align}
    \partial_\mu \phi(\bm{x})
    ,\quad
    \partial_\mu := \dfrac{\partial}{\partial x^\mu}
    \quad
    (\mu = 0,\dots,d)
    ,\quad
    x^0 := c t
    \quad
    \text{(c: the speed of light in vacuum)}
\end{align}
で書かれたLagrangian $L$を用意しよう。
力学自由度が空間の一点に局在している質点とは異なり、
場は空間的に広がりを持った自由度であるので
その相互作用の仕方も質点の場合とは大きく異なる。
物理学では経験的事実として
\begin{enumerate}
    \item{causality(因果律):原因より先に結果が起こってはいけない}
    \item{locality(局所性):遠く離れた地点で起こった事象が、直ちに他の地点の事象に影響を与えてはいけない。
    つまり、情報が速度無限大で伝わってはいけない}
\end{enumerate}
を仮定するので、
ある時刻$t$の物理量と別の時刻$t-t_0$とが直接相互作用すること、つまり
Lagrangianに$\phi(t) \psi(t-t_0)$のような項が現れることを禁止する。
Lagrangianに、例えば
\begin{align}
    \sqrt{\left(\dfrac{
        m c
    }{
        \hbar
    }\right)^2 + A^{\mu \nu}\partial_\mu \partial_\nu}
    \ \phi(t,\bm{x})
&=
    \dfrac{ m c }{ \hbar }
    \sqrt{
        1 + \left(\dfrac{ \hbar }{ m c }\right)^2
        A^{\mu \nu}\partial_\mu \partial_\nu
    }
    \ \phi(t,\bm{x})
\notag\\&= \dfrac{ m c }{ \hbar }
    \sum_{n=0}^\infty
    c_n
    (A^{\mu \nu}\partial_\mu \partial_\nu)^n
    \ \phi(t,\bm{x})
\notag\\&
\text{($A^{\mu\nu}$: a dimensionless matrix,  $c_n$: constants)}
\end{align}
のような無限階の微分を含めることも
localityを破るため禁止する。
operatorの関数がTaylor展開で与えられていた事を思い出そう。

さて、上のようなlocalityの制約により、
Lagrangianは時空上$\mathcal{M}^{1,d}$の
各点$(t, \bm{x})$の情報の和ないし積分で書ける筈である。
この局所的な情報をLagrangian密度と言い、
$\mathcal{L}$と書く。
当然Lagrangianは$L := \int d^d x \mathcal{L}$、
作用は$S= \int dt L = \int d^D x \mathcal{L}$である。
場の変分$\phi \mapsto \phi + \delta \phi$のもとで
作用の変分は
\begin{align}
    \delta S
    &=
    \int d^D x
    \left[
        \mathcal{L}(\phi + \delta \phi ,
        \partial \phi + \partial (\delta \phi))
        - \mathcal{L}
    \right]
\notag\\&=
    \int d^D x
    \left[
        \delta \phi
        \dfrac{\delta \mathcal{L}}{\delta \phi}
        +
        \partial_\mu (\delta \phi)
        \dfrac{\delta \mathcal{L}}{\delta (\partial_\mu \phi)}
    \right]
\notag\\&=
    \int d^D x
    \ 
    \delta \phi
    \left[
        \dfrac{\delta \mathcal{L}}{\delta \phi}
        -
        \partial_\mu
        \dfrac{\delta \mathcal{L}}{\delta (\partial_\mu \phi)}
    \right]
    + \text{(surface term)}
\end{align}
であり、作用の停留条件$\delta S = 0$から
場のEuler-Lagrange方程式は
\begin{align}
    0 &=
    \dfrac{\delta \mathcal{L}}{\delta \phi}
    -
    \partial_\mu
    \dfrac{\delta \mathcal{L}}{\delta (\partial_\mu \phi)}
\label{field EL eq}
\end{align}
と導かれる。

\subsubsection{自由scalar場の例}

$\phi(t, \bm{x}), \varphi(t, \bm{x})$を
それぞれ実および複素scalar場とし、
Lagrangianを
\begin{align}
    \mathcal{L}_{\rm real}
    &:=
    -
    \dfrac{ \eta^{\rho\sigma} }{2}
        \partial_\rho \phi
        \partial_\sigma \phi
    -
    \dfrac{m^2}{2} \phi^2
\\
    \mathcal{L}_{\rm complex}
    &:=
    -
    \eta^{\rho\sigma}
        \partial_\rho \varphi^*
        \partial_\sigma \varphi
    -
    m^2 |\varphi|^2
\end{align}
で与える。
ここで$\eta$は無次元の定数行列である。

これらの運動方程式(\ref{field EL eq})は
d'Alembert operator
(ダランベール演算子、
d'Alembertian、ダランベルシアン、
古い文献ではd'Alemberian、ダランベーリアンとも)
$\Box := \eta^{\mu \sigma}
    \partial_\mu
    \partial_\sigma$
を用いて
\begin{align}
    0 &=
    \dfrac{\delta \mathcal{L}_{\rm real}}
        {\delta \phi}
    -
    \partial_\mu
    \dfrac{\delta \mathcal{L}_{\rm real}}
        {\delta (\partial_\mu \phi)}
    =
    - \dfrac{m^2}{2}2 \phi
    -
    \partial_\mu
    \left[
        -
        \dfrac{ \eta^{\rho \sigma} }{2}
        \left(
            \delta_\rho^\mu
            \partial_\sigma \phi
        +
            \partial_\rho \phi
            \delta_\sigma^\mu
        \right)
    \right]
\notag\\&
    =
    - m^2 \phi
    +
    \partial_\mu
    \left[
        \dfrac{
            \eta^{\mu \sigma}
            \partial_\sigma \phi
        +
            \eta^{\rho \mu}
            \partial_\rho \phi
        }{2}
    \right]
    =
    - m^2 \phi
    +
    \eta^{\mu \sigma}
        \partial_\mu
        \partial_\sigma
    \phi
\notag\\&
    =
    (\Box - m^2) \phi
\end{align}
および
\begin{subequations}
\begin{align}
    0 &=
    \dfrac{\delta \mathcal{L}_{\rm complex}}
        {\delta \varphi^*}
    -
    \partial_\mu
    \dfrac{\delta \mathcal{L}_{\rm complex}}
        {\delta (\partial_\mu \varphi^*)}
    =
    -
    m^2 \varphi
    -
    \partial_\mu
    \left(
        -
        \eta^{\rho\sigma}
        \delta_\rho^\mu
        \partial_\sigma \varphi        
    \right)
\notag\\&
    =
    -
    m^2 \varphi
    +
    \eta^{\mu\sigma}
        \partial_\mu
        \partial_\sigma
        \varphi
    =
    (\Box - m^2)
    \varphi
\\
    0 &=
    \dfrac{\delta \mathcal{L}_{\rm complex}}
        {\delta \varphi}
    -
    \partial_\mu
    \dfrac{\delta \mathcal{L}_{\rm complex}}
        {\delta (\partial_\mu \varphi)}
    =
    (\Box - m^2)
    \varphi^*
\end{align}
\end{subequations}
と得られる。

\subsubsection{Massive自由Vector場の例}

$\eta_{\mu\nu}$を無次元の定数対称行列、
$\eta^{\mu\nu}$をその逆行列
$\eta_{\mu\lambda} \eta^{\lambda \nu} = \delta_\mu^\nu$として、
vector場$A^\mu$に対し
$A_\mu := \eta_{\mu\nu} A^\nu$
および
$F_{\mu\nu} :=
\partial_\mu A_\nu
- \partial_\nu A_\mu$
を定義する。
$F_{\mu\nu}$を場の強さテンソル
(field strength tensor)
といい、
以下では同様に添え字の上げ下げを$\eta$で行う。
Lagrangianを
\begin{align}
    \mathcal{L}_{\rm massvec}
    :=
    -
    \dfrac{1}{4}
        \eta^{\mu\rho}
        \eta^{\nu\sigma}
    F_{\mu\nu} F_{\rho\sigma}
    -
    \dfrac{m^2}{2}
        \eta_{\mu \nu}
        A^\mu A^\nu
=
    - \dfrac{1}{4}
    F_{\mu\nu} F^{\mu\nu}
    -
    \dfrac{m^2}{2}
        A^\mu A_\mu
\end{align}
で与えるとき、
\begin{align}
    \dfrac{\delta}
    {\delta (\partial_\sigma A_\rho)}
    F_{\mu\nu}
    &=
    \dfrac{\delta}
    {\delta (\partial_\sigma A_\rho)}
    \left[
        \partial_\mu A_\nu
        - \partial_\nu A_\mu
    \right]
    =
    \delta_\mu^\sigma
    \delta_\nu^\rho
    -
    \delta_\nu^\sigma
    \delta_\mu^\rho
\\
    F_{\mu\nu} &= - F_{\nu\mu}
\end{align}
に気を付けると
Euler-Lagrange方程式(\ref{field EL eq})は
\begin{align}
    0
    &=
    \dfrac{\delta \mathcal{L}_{\rm massvec}}
        {\delta A_\rho}
    -
    \partial_\sigma
    \dfrac{\delta \mathcal{L}_{\rm massvec}}
        {\delta (\partial_\sigma A_\rho)}
    =
    -
    \dfrac{m^2}{2}
        \eta_{\mu \nu}
        \left(
            A^\mu \eta^{\nu\rho}
        +
            \eta^{\mu\rho} A^\nu
        \right)
    -
    \partial_\sigma
    \dfrac{\delta}
        {\delta (\partial_\sigma A_\rho)}
        \left[
            - \dfrac{1}{4}
            F_{\mu\nu} F^{\mu\nu}
        \right]
\notag\\&
    =
    -
    \dfrac{m^2}{2}
        \left(
            A^\mu \delta_\mu^{\rho}
        +
            \delta_{\nu}^\rho A^\nu
        \right)
    +
    \dfrac{1}{4}
    \partial_\sigma
    \bigg\{
        F_{\mu\nu}
        \left[
            \eta^{\mu \sigma}
            \eta^{\nu \rho}
        -
            \eta^{\nu\sigma}
            \eta^{\mu \rho}
        \right]
    +
        \left[
            \delta_\mu^\sigma
            \delta_\nu^\rho
            -
            \delta_\nu^\sigma
            \delta_\mu^\rho        
        \right]
        F^{\mu\nu}
    \bigg\}
\notag\\&
    =
    -
    m^2
    A^\rho
    +
    \dfrac{1}{4}
    \partial_\sigma
    \bigg\{
        F^{\sigma\rho}
    -
        F^{\rho\sigma}
    +
        F^{\sigma\rho}
    -
        F^{\rho\sigma}
    \bigg\}
=
    -
    m^2
    A^\rho
    +
    \partial_\sigma
        F^{\sigma\rho}
\notag\\&
=
    -
    m^2
    A^\rho
    +
    \Box A^\rho
    -
    \eta^{\rho \lambda}
    \partial_\sigma \partial_\lambda
    A^\sigma
=
    \bigg[
        \delta^\rho_\sigma
        (\Box - m^2)
    -
        \eta^{\rho \lambda}
        \partial_\sigma \partial_\lambda
    \bigg]
    A^\sigma
\end{align}
と得られる。

特に、$F_{\mu\nu}$は
gauge transformation(ゲージ変換)
$A'^\mu = A^\mu + \partial^\mu \Lambda$
のもとで不変
\begin{align}
    F'_{\mu\nu}
&=
    \partial_\mu A'_\nu
    -
    \partial_\nu A'_\mu
=
    \partial_\mu (A_\nu + \partial_\nu \Lambda)
    -
    \partial_\nu (A_\mu + \partial_\mu \Lambda)
\notag\\&
=
    \partial_\mu A_\nu
    -
    \partial_\nu A_\mu
    + (
        \partial_\mu \partial_\nu
    -
        \partial_\nu \partial_\mu
    )\Lambda
=
    F_{\mu\nu}
\end{align}
である事に注意しよう。
従って作用が$F_{\mu\nu}$のみを使って書かれる
massless $m^2 = 0$の場合のみ
作用および運動方程式はgauge symmetry
(ゲージ対称性)を持つ。
逆の言い方をすると、
理論の作用がgauge symmetryを持つよう要請する場合
vector場はmassを持てないことが分かる。

\subsubsection{Hamilton形式}

場の正準共役量(運動量と呼ばないのは、
これ以外に場の理論の意味でmomentumと呼ばれるべき適切な量、
つまりenergy-momentum tensor (\ref{energy-momentum})があるからだ)は
\begin{align}
    \pi(t, x) := \dfrac{\partial \mathcal{L}}{\partial \dot{\phi}}
\end{align}
で定義される。
Hamiltonianも解析力学の場合と全く同様に
\begin{subequations}
\begin{align}
    H
    &:=
    \int d^D x\ \mathcal{H}
\\
    \mathcal{H}
    &:=
    \pi(t, \bm{x}) \dot{\phi}(t, \bm{x})
    -
    \mathcal{L}(t, \bm{x})
    =
    m^2 \phi
    -
    \partial_\mu
    \left(
        - \eta^{\rho \sigma}
        \delta_\rho^\mu
        \partial_\sigma \phi
    \right)
\end{align}
\end{subequations}
と定義される。

\subsubsection{Energy-Momentum Tensor}

Noetherの定理は場の理論にも拡張できる。

時空並進に対応する保存量である
energy-momentum tensor(stress-energy tensor、stress tensor、エネルギー運動量テンソル)は
計量tensor $g_{\mu\nu}$を用いて
\begin{align}
    T^{\mu\nu}
    :=
    -
    \dfrac{2}{\sqrt{|g|} }
    \dfrac{\partial L_{\mathrm{matter}}}{\partial g_{\mu\nu}}
\label{energy-momentum}
\end{align}
で定義される。
ただし、一般相対論まで行くと重力場そのものの作用も
気にしなければならないので、
今はLagrangianの中に重力を表すEinstein-Hilbert項を含めていないと示すため
$L_{\mathrm{matter}}$と書いた。

\subsubsection{Free Scalar Fieldの例}

$n$次元対角行列$A$の$(i,i)$成分が$a_i$であるとき
\begin{align}
    A = \mathrm{diag}(a_1, a_2,\dots)
    =
    \begin{pmatrix}
        a_{1}  & \cdots & 0      & \cdots & 0
        \\
        \vdots & \ddots &        &        & \vdots
        \\
        0      &        & a_{i}  &        & 0
        \\
        \vdots &        &        & \ddots & \vdots
        \\
        0      & \cdots & 0      & \cdots & a_{n}
    \end{pmatrix}
\end{align}
と書こう。
$D$次元Minkowski空間の
計量(metric tensor)を
\begin{align}
    \eta_{\mu\nu}
    :=
    \mathrm{diag}(-1, +1, +1, \dots)
\end{align}
と定義しておく。
他にも
\begin{align}
    \eta_{\mu\nu}
    :=
    \mathrm{diag}(+1, -1, -1, \dots)
\end{align}
とする流儀があるため、
前者をEast coast sign conventionとか
mostly plus signsと言い、
後者のWest coast conventionまたは
mostly minuses conventionと区別することがある。
複数のconventionがあるのは名前の通り
太古のAmericaで東海岸と西海岸の間に十分な交流がなかったせいで、
$(3+1)$次元時空に限って議論すればよい現象論屋の間では
後者が未だに隆盛を誇っているようであるが、
mostly minusでは
Euclid空間への移行$\eta_{\mu\nu} \to \delta_{\mu\nu}$に不便であるし、
$\det g$の符号が次元によって変わってしまうのも
一般次元への拡張のためにかなり都合が悪いので、
数理物理に手を出す人間はmostly plusの計量を用いることが多いようだ。
物理的に表す内容は全く同じなので好きな方を使えばよいのだが、
筆者個人としてはmostly minusなどという歴史上の遺物に慣れてしまう前に
\href{http://www.math.columbia.edu/~woit/wordpress/?p=7773}
{The West Coast Metric is the Wrong One}
なども参照することを勧める。

\subsection{場の量子化}

古くは第2量子化と呼ばれる事もあったが、
量子力学に何らかの操作を施して新たな量子論を得るのではなく、
むしろ場の量子論の非相対論的極限で量子力学が現れるという理解の方が正しいため、
この用語は廃れてしまった。
同様に、かつては単純に量子力学を相対論的に拡張しようとした
相対論的量子力学という試みもあったが、
負の確率が現れてしまうなど様々な困難を孕んでおり、
それらは場の量子論の視点からは自然に解決するため
今更それを学ぶ価値はそう高くないだろう。

\subsubsection{非相対論的なScalar場}

正準交換関係
\begin{align}
    [\hat{\phi}(\bm{x}), \hat{\pi}(\bm{y})]
    &=
    i \hbar \delta(\bm{x} - \bm{y})
\end{align}
を課す。

\subsubsection{非相対論的なSpinor場}

\subsubsection{相対論的に可能な場}
\label{representation of Lorentz group}

\section{摂動展開と繰り込み}

\subsection{一般の相互作用が満たすべき性質}
Wightman axioms、Haag's theorem、Coleman–Mandula theorem、Haag-\L{}opusza\'nski-Sohnius theorem、
locality
\label{locality}
、
cluster decomposition principle