\section{自由場の量子論}
\label{sec: QFT}

場の量子論と言うとFeynman diagram
(古くはgraphと呼ぶ文献もある)を使った計算に
皆さん憧れているのだろうが、
これは摂動計算に現れるものである。
摂動論は問題を解ける部分とそこからのズレに分けて
解析するというものであったので、
とにかく解ける理論がないと話にならない。
可解な場の量子論もそれ自身興味深い話題ではあるものの、
以下では解ける模型として非相対論的な場合から始め、
相対論的な場まで含めた自由場の量子化を扱う。

\subsection{自由場の古典論、場の解析力学}

量子化するというからには
量子化される対象となる古典論を用意する必要がある。
適当な場$\phi(\bm{x})$と
その空間微分$\partial_i \phi(\bm{x})$
で書かれたLagrangian $L$を用意しよう。
空間$d$次元、時間$1$次元の$D:= d+1$次元時空を考え、
Lagrangian密度を$\mathcal{L}$と書く。
当然Lagrangianは$L := \int d^d x \mathcal{L}$である。
場の変分$\phi \mapsto \phi + \delta \phi$のもとで
作用の変分は
\begin{align}
    \delta S
    &=
    \int d^{d+1} x
    \left[
        \mathcal{L}(\phi + \delta \phi ,
        \partial \phi + \partial (\delta \phi))
        - \mathcal{L}
    \right]
\notag\\&=
    \int d^{d+1} x
    \left[
        \delta \phi
        \dfrac{\delta \mathcal{L}}{\delta \phi}
        +
        \partial (\delta \phi)
        \dfrac{\delta \mathcal{L}}{\delta (\partial\phi)}
    \right]
\end{align}


\subsection{場の量子化}

古くは第2量子化と呼ばれる事もあったが、
量子力学に何らかの操作を施して新たな量子論を得るのではなく、
むしろ場の量子論の非相対論的極限で量子力学が現れるという理解の方が正しいため、
この用語は廃れてしまった。
同様に、かつては単純に量子力学を相対論的に拡張しようとした
相対論的量子力学という試みもあったが、
負の確率が現れてしまうなど様々な困難を孕んでおり、
それらは場の量子論の視点からは自然に解決するため
今更それを学ぶ価値はそう高くないだろう。

\subsubsection{非相対論的なScalar場}

\subsubsection{非相対論的なSpinor場}

\subsubsection{相対論的に可能な場}

\section{摂動展開と繰り込み}
