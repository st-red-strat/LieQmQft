\section{古典電磁気学}

\subsection{Maxwell方程式}

古典電磁気学によると、
電荷密度$\rho$および
電流密度$\bm{j}$の存在下で
電場(electric field、
工学の分野では電界とも)
$\bm{E}$
および
磁場(magnetic field、磁束密度、
工学では磁界とも)
$\bm{B}$は
Maxwell方程式
\begin{subequations}
\begin{align}
&\text{電場に関するGaussの法則}:
  &&\nabla \cdot \bm{E}
=
  \frac{\rho}{\epsilon_0}
\label{eq:maxwell-divE}
\\
&\text{磁場に関するGaussの法則}:
  &&\nabla \cdot \bm{B}
= 0
\label{eq:maxwell-divB}
\\
&\text{Faradayの電磁誘導の法則}:
  &&\nabla \times \bm{E}
=
  - \frac{\partial \bm{B}}
    {\partial t}
\label{eq:maxwell-rot E}
\\
&\text{Amp\'ereの法則}:
  &&\nabla \times \bm{B}
=
  \mu_0 \bm{j}
  +
  \mu_0 \epsilon_0
    \frac{\partial \bm{E}}
      {\partial t}
\label{eq:maxwell-rot B}
\end{align}
\end{subequations}
に従う。
定数
$\epsilon_0, \mu_0$をそれぞれ
真空中の誘電率、真空中の透磁率と呼ぶ。
なお、電磁気学においては
媒質中でないこと、
またはそれに加えて
$\rho = 0, \bm{j} = \bm{0}$
であることを指して真空という。

\subsubsection{連続の式と電荷保存則}

Maxwell方程式を組み合わせ、
滑らかな関数に対して
偏微分の順序を入れ替えてよい事
および公式
(\ref{div rot = 0})を使うと
\begin{align}
  \dfrac{\partial \rho}{\partial t}
+
  \nabla \cdot \bm{j}
=
  \epsilon_0
  \nabla \cdot
  \dfrac{\partial \bm{E}}{\partial t}
+
  \dfrac{1}{\mu_0}
  \nabla \cdot
  \left[
    \nabla \times \bm{B}
  -
    \mu_0 \epsilon_0
    \frac{\partial \bm{E}}
      {\partial t}
  \right]
= 0
\label{electric charge conservation}
\end{align}
が示せる。
この方程式はある点における電荷密度の時間変化
$\dfrac{\partial \rho}{\partial t}$
が、その点に流入する電流
$- \nabla \cdot \bm{j}$
(数学では普通ある領域の境界(これは閉曲面となる)
における法線vectorについて、
領域に対して外向きを正と取る。
ここではそれに合わせて
電流が流入ではなく流出する向きを正に取ったため、
先頭に負号が現れている)
に等しいこと、つまり電荷の保存則
(charge conservation、
the law of conservation of charge、
連続の式(equation of continuity)とも)
を表している。

\subsubsection{Poincar\'eの補題と
  Scalar Potential、
  Vector Potential}

Poincar\'eの補題により
(\ref{eq:maxwell-divB})は
(少なくとも局所的に)
vector potentialの存在
\begin{align}
  {}^\exists
  \bm{A}
\quad
\text{ s.t. }
\quad
  \nabla \times \bm{A}
=
  \bm{B}
\end{align}
を保証し、
逆に$\bm{A}$を使う限り
(\ref{eq:maxwell-divB})
$\nabla \cdot \bm{B}
= \nabla \cdot
(\nabla \times \bm{A})
= 0$
はベクトル解析の恒等式
(\ref{div rot = 0})から常に成り立つ。

次に
(\ref{eq:maxwell-rot E})は
$\bm{A}$を使って
\begin{align}
  0 =
  \nabla \times \bm{E}
  +
  \frac{\partial \bm{B}}
    {\partial t}
=
  \nabla \times
  \left(
    \bm{E}
    +
    \frac{\partial \bm{A}}
      {\partial t}      
  \right)
\end{align}
となるので、再びPoincar\'eの補題により
scalar potential
\begin{align}
  {}^\exists \phi
\quad
\text{ s.t. }
\quad
  - \nabla \phi
=
  \bm{E}
  +
  \frac{\partial \bm{A}}
    {\partial t}
\end{align}
が局所的に存在し、
(\ref{eq:maxwell-rot E})
も
$\nabla \times
\left(
  \bm{E} +
  \dfrac{\partial \bm{A}}
    {\partial t}
\right)
= - \nabla \times \nabla \phi
= 0$
の形になるため
恒等式(\ref{rot grad = 0})
から自明に成り立つ。

まとめると、vector potential
$\bm{A}$および
scalar potential $\phi$
を用いて電磁場は
\begin{subequations}
\begin{align}
  \bm{E} (t, \bm{x})
&=
  - \nabla \phi (t, \bm{x})
  - \dfrac{\partial \bm{A}}
    {\partial t} (t, \bm{x})
\\
  \bm{B} (t, \bm{x})
&=
  \nabla \times
  \bm{A} (t, \bm{x})
\end{align}
\label{relation between potential and electric or magnetic field}
\end{subequations}
と書け、これらの従うMaxwell方程式は
公式
(\ref{rot rot = grad div - laplacian})
を使って
\begin{subequations}
\begin{align}
  - \frac{\rho}{\epsilon_0}
&=
  - \nabla \cdot \bm{E}
=
  - \nabla \cdot
  \left[
    - \nabla \phi (t, \bm{x})
    - \dfrac{\partial \bm{A}}
      {\partial t} (t, \bm{x})
  \right]
=
  \Delta \phi (t, \bm{x})
  +
    \nabla \cdot
    \dfrac{\partial \bm{A}}
      {\partial t} (t, \bm{x})
\\
  - \mu_0 \bm{j}
&=
  \mu_0 \epsilon_0
    \frac{\partial \bm{E}}
      {\partial t}
  -
  \nabla \times \bm{B}
=
  \mu_0 \epsilon_0
    \frac{\partial}{\partial t}
  \left[
    - \nabla \phi (t, \bm{x})
    - \dfrac{\partial \bm{A}}
      {\partial t} (t, \bm{x})
  \right]
  -
  \nabla \times
    (\nabla \times \bm{A})
\notag\\&
=
  -
  \mu_0 \epsilon_0
  \left[
    \nabla
    \frac{\partial \phi (t, \bm{x})}
      {\partial t}
  +
    \dfrac{\partial^2 \bm{A}}
      {\partial t^2} (t, \bm{x})
  \right]
  -
  \left[
    \nabla
      (\nabla \cdot \bm{A} (t, \bm{x}))
  -
    \Delta \bm{A} (t, \bm{x})
  \right]
\notag\\&
=
  \left[
    - \mu_0 \epsilon_0
    \dfrac{\partial^2}{\partial t^2}
  +
    \Delta
  \right]
    \bm{A} (t, \bm{x})
  -
  \nabla
  \left[
    \mu_0 \epsilon_0
    \frac{\partial \phi (t, \bm{x})}
      {\partial t}
  +
    \nabla \cdot
      \bm{A} (t, \bm{x})
  \right]
\end{align}
\label{maxwell eq of potentials}
\end{subequations}
の$2$本に集約される。

potentialから再び電場や磁場を得るには
単に(\ref{relation between potential and electric or magnetic field})
に代入すればよいのだが、
特に公式
(\ref{3rd order epsilon to delta2 formula})
に気を付けて磁場を成分で書くと
\begin{subequations}
\begin{align}
  & F_{jk} :=
  \partial_j A_k - \partial_k A_j
=
  (\delta_{jn} \delta_{km}
  - \delta_{kn} \delta_{jm})
  \partial_n A_m
=
  \epsilon_{ijk} \epsilon_{inm}
  \partial_n A_m
=
  \epsilon_{ijk} B_i
\\
  & \qquad B_i =
  (\nabla \times A)_i =
  \epsilon_{ijk} \partial_j A_k
=
  \dfrac{1}{2}
  \epsilon_{ijk}
  ( \partial_j A_k
  - \partial_k A_j )
=
  \dfrac{1}{2}
  \epsilon_{ijk}
  F_{jk}
\end{align}
\label{field strength and magnetic field}
\end{subequations}
という相互関係がある事が分かる。
この相互関係の一般論は(\ref{3d hodge and axial vector})を、
$F_{ij}$については(\ref{field strength definition})
を参照のこと。

\subsubsection{擬vectorとしての磁場$\bm{B}$と
時間反転対称性}

磁場とpotentialの関係は
(\ref{field strength and magnetic field})で表され、
これを(\ref{3d hodge and axial vector})
と見比べる限りでは磁場が擬vectorであることを
自然に受け入れてしまいそうになる。
しかし磁場$\bm{B}$はそれ自身が観測可能な物理量であって、
vector potential $\bm{A}$を使って定義されている訳ではない。
それどころか、例えば空間に無限に長いsolenoidを置いて磁場の定義域を単連結でなくした場合のように、
物理的な問題設定によっては必ずしもPoincar\'eの補題が成り立たない場合もある。
このように与えられた磁場に対してvector potentialは存在するとは限らない量であるので、
vector potentialと完全反対称tensorを用いた磁場の表式が存在することは
磁場が擬vectorであることの説明としては全く不十分である。

ここでは、電磁場中にあってLorentz力を受ける点電荷の従うNewtonの運動方程式
(\ref{lorentz force for charged point particle})
を用いて磁場の空間反転の下での性質を議論してみよう。
まず電磁場のないごく普通のNewtonの運動方程式(\ref{newton eom})は
\begin{align}
  \bm{F} = m \ddot{ \bm{x} }
\end{align}
という形をしており、
時間を反転する操作
$t \mapsto t' := - t$のもとで
\begin{align}
  \bm{x}(t)
  \mapsto \bm{x}(t')
,\qquad
  \dot{ \bm{x} }(t)
  \mapsto - \dot{ \bm{x} }(t')
,\qquad
  \ddot{ \bm{x} }(t)
  \mapsto \ddot{ \bm{x} }(t')
\end{align}
という変換則に気を付けると
変換後も方程式は不変
\begin{align}
  \bm{F}(t') = m \ddot{ \bm{x} }(t')
\end{align}
であることが分かる。
電磁場を含む運動方程式も同様に
時間反転の下で不変であることが期待されるが、
電荷$+q$の点電荷が受けるLorentz力は
\begin{align}
  q \Big[
    \bm{E}
  +
   \dot{\bm{x}}
    \times
    \bm{B}
  \Big]
&\qquad \mapsto \qquad
\end{align}
のように変換するので

\subsection{静電磁気学(electrostatics)とGreen関数}

力学系や方程式系が時刻$t$に依存しない場合を
静的(static、time-independent、定常とも)という。
静電磁場の従う方程式を得るには
単に
(\ref{maxwell eq of potentials})
から時間微分を含む項を落とせばよく
\begin{subequations}
\begin{align}
  - \frac{\rho}{\epsilon_0}
&=
  \Delta \phi
\label{static eq for ele-mag scalar potential}
\\
  - \mu_0 \bm{j}
&=
  - [
    \nabla
    ( \nabla \cdot \bm{A} )
  -
    \Delta \bm{A}
  ]
=
  - \nabla \times
    (\nabla \times \bm{A})
\label{static eq for ele-mag vector potential}
\end{align}
\end{subequations}
となる。
静的な場合には
電場の方程式と磁場の方程式が
完全に分離しており、
それぞれ独立に解くことが可能であることに
注意しよう。

$3$次元LaplacianのGreen関数
$ - \dfrac{1}{4 \pi r} $
を
(\ref{green function of n dim laplacian})
で求めたので、
公式
(\ref{convolution representation of green function solution})
に代入することで
(Green関数により決まる、
無限遠で十分早く$0$となるという境界条件のもと)
静電場のscalar potentialは
\begin{align}
  \phi (\bm{x})
&=
  \int d^3 x'
  \left[
    - \dfrac{1}{
      4 \pi |\bm{x} - \bm{x}'|
    }
  \right]
  \left[
    - \frac{\rho(\bm{x}')}{\epsilon_0}
  \right]
=
  \int d^3 x'
    \dfrac{\rho(\bm{x}')}{
      4 \pi \epsilon_0
      |\bm{x} - \bm{x}'|
    }
\end{align}
と書けることが分かる。
実際、例えば電荷$q$を持ち
位置$\bm{y}$に固定された点電荷
$\rho(\bm{x})
= q \delta(\bm{x} - \bm{y})$
の場合にこの解は良く知られた
Coulomb potential
\begin{align}
  \phi (\bm{x})
&=
  \int d^3 x'
    \dfrac{
      q \delta(\bm{x}' - \bm{y})
    }{
      4 \pi \epsilon_0
      |\bm{x} - \bm{x}'|
    }
=
    \dfrac{ q }{
      4 \pi \epsilon_0
      |\bm{x} - \bm{y}|
    }
\label{coulomb potential of point charge}
\end{align}
を確かに再現する。

一方で、
vector potentialの従う方程式
(\ref{static eq for ele-mag vector potential})
を解くためには、第$1$項を落とした方程式
\begin{align}
  \mu_0 \bm{j}
&=
  -
    \Delta
    \bm{C}
\end{align}
がscalar potentialと同様にすると
簡単に解けて、
解が
\begin{align}
  \bm{C} (\bm{x})
&=
  \int d^3 x'
  \left[
    - \dfrac{1}{
      4 \pi |\bm{x} - \bm{x}'|
    }
  \right]
  \left[
    - \mu_0
    \bm{j} (\bm{x}')
  \right]
=
  \int d^3 x'
    \dfrac{
      \mu_0 \bm{j} (\bm{x}')
    }{
      4 \pi
      |\bm{x} - \bm{x}'|
    }
\end{align}
となる事を使うとよい。
$\bm{x}$に対する微分を$\nabla$、
$\bm{x}'$に対する微分を$\nabla'$と書くと、
部分積分により
\begin{align}
  \nabla \cdot \bm{C}(\bm{x})
&=
  \int d^3 x'
    \nabla \cdot
    \dfrac{
      \mu_0 \bm{j} (\bm{x}')
    }{
      4 \pi
      |\bm{x} - \bm{x}'|
    }
=
  \int d^3 x'
    \bm{j} (\bm{x}')
  \cdot
    \nabla
    \dfrac{ \mu_0 }{
      4 \pi
      |\bm{x} - \bm{x}'|
    }
\notag\\&
=
  \int d^3 x'
    \bm{j} (\bm{x}')
  \cdot
    \nabla'
    \dfrac{ -\mu_0 }{
      4 \pi
      |\bm{x} - \bm{x}'|
    }
=
  \int d^3 x'
    \dfrac{ \mu_0 }{
      4 \pi
      |\bm{x} - \bm{x}'|
    }
    \nabla' \cdot \bm{j} (\bm{x}')
\end{align}
となるが、
電荷の保存則
(\ref{electric charge conservation})
は静的な場合には単に
$\nabla \cdot \bm{j} = 0$
であるので$\nabla \cdot \bm{C} = 0$
が分かる。
すなわち$\bm{C}$は実は
vector potentialの満たすべき方程式
(\ref{static eq for ele-mag vector potential})
を満たしており、
静的な電流の作る磁場は
\begin{align}
  \bm{A} (\bm{x})
&=
  \bm{C} (\bm{x})
=
  \int d^3 x'
    \dfrac{
      \mu_0 \bm{j} (\bm{x}')
    }{
      4 \pi
      |\bm{x} - \bm{x}'|
    }
\end{align}
と求まるのである。
これはCoulomb gauge
$\nabla \cdot\bm{A} = 0$
を取ったことに相当する。
実際、
この解はよく知られた
Biot-Savart(ビオ・サバール)の法則
\begin{align}
  \bm{B}
&=
  \nabla \times \bm{A}
=
  \int d^3 x'
  \nabla \times
    \dfrac{
      \mu_0 \bm{j} (\bm{x}')
    }{
      4 \pi
      |\bm{x} - \bm{x}'|
    }
=
  \int d^3 x'
  \bm{j} (\bm{x}') \times
  \nabla
    \dfrac{
      - \mu_0
    }{
      4 \pi
      |\bm{x} - \bm{x}'|
    }
% \notag\\&
=
  \int d^3 x'
  \bm{j} (\bm{x}') \times
    \dfrac{
      \mu_0
      (\bm{x} - \bm{x}')
    }{
      4 \pi
      |\bm{x} - \bm{x}'|^3
    }
\end{align}
を再現する。
\footnote{
  あるいは
  $\bm{j}(\bm{x}')$に平行で同じ向きの
  線素$d \bm{l}'$および
  それに垂直な微小曲面$dS'$に対応する
  面素$d \bm{S}'$を用いて
  $d^3 x' = d \bm{l}' \cdot d \bm{S}'$
  かつ
  $I = \int_{S'} d \bm{S}' \cdot \bm{j}(\bm{x}')$
  ($I$は面$S'$を貫く電流)
  に注意すると、微小形で
  \begin{align}
    d \bm{B}
  &=
    \left[
      d \bm{l}' \cdot
      \int_{S'} d \bm{S}'
    \right]
    \nabla \times
      \dfrac{
        \mu_0 \bm{j} (\bm{x}')
      }{
        4 \pi
        |\bm{x} - \bm{x}'|
      }
  =
    \left[
      d \bm{l}' \cdot
      \int_{S'} d \bm{S}'
    \right]
    \bm{j} (\bm{x}') \times
      \dfrac{
        \mu_0
        (\bm{x} - \bm{x}')
      }{
        4 \pi
        |\bm{x} - \bm{x}'|^3
      }
  \notag\\&
  =
    \int_{S'}
      [d \bm{S}' \cdot \bm{j} (\bm{x}')]
    d \bm{l}' \times
      \dfrac{
        \mu_0
        (\bm{x} - \bm{x}')
      }{
        4 \pi
        |\bm{x} - \bm{x}'|^3
      }
  =
    I d \bm{l}'
    \times
      \dfrac{
        \mu_0
        (\bm{x} - \bm{x}')
      }{
        4 \pi
        |\bm{x} - \bm{x}'|^3
      }
  \end{align}
  と書いても同じである。
}

\subsubsection{真空中の電磁波解}

真空中$\rho = \bm{j} = 0$の
Maxwell方程式
\begin{subequations}
\begin{align}
  \nabla \cdot \bm{E}
&=
  0
\\
  \nabla \cdot \bm{B}
&= 0
\\
  \nabla \times \bm{E}
&=
  - \frac{\partial \bm{B}}
    {\partial t}
\\
  \nabla \times \bm{B}
&=
  \mu_0 \epsilon_0
    \frac{\partial \bm{E}}
      {\partial t}
\end{align}
\end{subequations}
または
\begin{subequations}
\begin{align}
  0
&=
    \Delta \phi (t, \bm{x})
  +
    \nabla \cdot
    \dfrac{\partial \bm{A}}
      {\partial t} (t, \bm{x})
\\
  0
&=
  \left[
    - \mu_0 \epsilon_0
    \dfrac{\partial^2}{\partial t^2}
  +
    \Delta
  \right]
    \bm{A} (t, \bm{x})
  -
  \nabla
  \left[
    \mu_0 \epsilon_0
    \frac{\partial \phi (t, \bm{x})}
      {\partial t}
  +
    \nabla \cdot
      \bm{A} (t, \bm{x})
  \right]
\end{align}
\label{maxwell eq of potentials in vacuum}
\end{subequations}
を考えよう。
