\appendix
\renewcommand{\theequation}{A.\arabic{section}.\arabic{equation}}
\setcounter{equation}{0}

\section{量子力学の公式}

\subsection{交換関係・反交換関係の基本的な性質}

証明は読者の演習問題とする。
\begin{align}
    [\hat{A}, \hat{B}] &= - [\hat{B}, \hat{A}]
\\
    [\hat{A}, \hat{B}\hat{C}]
   &=
   \hat{B}[\hat{A}, \hat{C}]
+
    [\hat{A}, \hat{B}] \hat{C}
\label{A,BC to B(A,C) + (A,B)C}
\\
    [\hat{A}\hat{B}, \hat{C}]
   &=
   \hat{A}\{\hat{B}, \hat{C}\}
-
    \{\hat{A}, \hat{C}\} \hat{B}
\\
    \left(\hat{A}\hat{B}\right)^\dagger
    &=
    \hat{B}^\dagger\hat{A}^\dagger
\\
    [\hat{A}, \hat{B}]^\dagger
    &=
    [\hat{B}^\dagger, \hat{A}^\dagger]
\end{align}

Baker-Campbell-Hausdrff formulaは
\begin{align}
    e^A e^B
    &=
    \exp\left(
        A + B
        + \dfrac{1}{2}[A,B]
        + \dfrac{1}{12}[A,[A,B]]
        - \dfrac{1}{12}[B,[A,B]]
        \dots
    \right)
\label{BCH formula}
\end{align}
であり、特に交換関係が$c$-数$[A,B]=c$の場合は
交換子を$2$回以上取ると必ず消えるため
\begin{align}
    e^A e^B
    &=
    \exp\left(
        A + B
        + \dfrac{1}{2}c
    \right)
\label{simpler BCH formula}    
\end{align}
となる。

正準変数$[\hat{q}_i, \hat{p}_j] = i\hbar\delta_{ij}$
については興味深い事実が成り立つ。
$C_n := \dfrac{1}{i\hbar} [\hat{q}^n_i, \hat{p}_j]$について
\begin{align}
    C_{n+1}
    &=
    \dfrac{1}{i\hbar}
    [\hat{q}^{n+1}_i, \hat{p}_j]
    =
    \hat{q}_i 
    \dfrac{1}{i\hbar}
    [\hat{q}_i^n, \hat{p}_j]
    +
    \dfrac{1}{i\hbar}
    [\hat{q}_i, \hat{p}_j] \hat{q}_i^n
    =
    \hat{q}_i C_n
    +
    \delta_{ij} \hat{q}_i^n
\end{align}
なる漸化式が導けるが、
これは初期条件
$C_0  = 0, C_1 = \delta_{ij}$のもとで
容易に
\begin{align}
    C_{n+1} &= 
    \dfrac{1}{i\hbar}
    [\hat{q}^{n+1}_i, \hat{p}_j]
    = \delta_{ij} (n+1) \hat{q}^n
\label{differential by commutator}
\end{align}
と解け、多項式の微分と同じ振る舞いを与える。
全く同様に
$\dfrac{1}{i\hbar}
[\hat{q}_i, \hat{p}^{n+1}_j]
= \delta_{ij} (n+1) \hat{p}^n$
も示される。
一般にoperatorの関数$F$の定義
(\ref{function of operator})
はTaylor展開で与えられていたので、
\begin{align}
    \dfrac{1}{i\hbar}
    [F(\{ \hat{q} \},\{ \hat{p} \}), \hat{p}_i]
    &=
    \dfrac{
        \partial F(\{ \hat{q} \},\{ \hat{p} \})
    }{
        \partial q_i
    }
\\
    \dfrac{1}{i\hbar}
    [F(\{ \hat{q} \},\{ \hat{p} \}), \hat{p}_j]
    &=
    \dfrac{
        \partial F(\{ \hat{q} \},\{ \hat{p} \})
    }{
        \partial p_j
    }
\end{align}
なる公式が得られる。

任意のoperator $\hat{O}$に対し、
そのHeisenberg表示(Heisenberg描像、Heisenberg picture)を
\begin{align}
    \hat{O}(t)
    :=
    \exp\left(
        -\dfrac{\hat{H} t}{i\hbar}
    \right)
        \hat{O}
    \exp\left(
        +\dfrac{\hat{H} t}{i\hbar}
    \right)
\end{align}
のように定義する。
$\hat{H}$のSchr\"odinger描像とHeisenberg描像は
一致する。
また、もちろんSchr\"odinger描像で
$\hat{O}$自身が顕わに$t$に依存している場合も
Heisenberg描像は同様に定義できる。
$\hat{H}$との交換関係は
\begin{align}
    [F(\{ \hat{q} \},\{ \hat{p} \}, t), \hat{H}]
    \label{Hamiltonian as time translation generator}
\end{align}
これはPoisson括弧で書かれた正準方程式(\ref{Hamilton e.o.m. in Poisson bracket})
ないし任意の関数$F$の時間発展(\ref{time evolution in Poisson bracket})
と同一の構造であり、
量子化とは
$\{A, B\}_{ \mathrm{P} }$
を
$\dfrac{1}{i\hbar} [\hat{A}, \hat{B}]$
で置き換える操作である、という
Bohrの対応原理(correspondence principle)をある意味で正当化する。
